% -----------------------------------------------------------------------------
% The MIT License (MIT)
%
% Copyright (c) 2015 Pejman Ghorbanzade
%
% Permission is hereby granted, free of charge, to any person obtaining a copy
% of this software and associated documentation files (the "Software"), to deal
% in the Software without restriction, including without limitation the rights
% to use, copy, modify, merge, publish, distribute, sublicense, and/or sell
% copies of the Software, and to permit persons to whom the Software is
% furnished to do so, subject to the following conditions:
%
% The above copyright notice and this permission notice shall be included in
% all copies or substantial portions of the Software.
%
% THE SOFTWARE IS PROVIDED "AS IS", WITHOUT WARRANTY OF ANY KIND, EXPRESS OR
% IMPLIED, INCLUDING BUT NOT LIMITED TO THE WARRANTIES OF MERCHANTABILITY,
% FITNESS FOR A PARTICULAR PURPOSE AND NONINFRINGEMENT. IN NO EVENT SHALL THE
% AUTHORS OR COPYRIGHT HOLDERS BE LIABLE FOR ANY CLAIM, DAMAGES OR OTHER
% LIABILITY, WHETHER IN AN ACTION OF CONTRACT, TORT OR OTHERWISE, ARISING FROM,
% OUT OF OR IN CONNECTION WITH THE SOFTWARE OR THE USE OR OTHER DEALINGS IN
% THE SOFTWARE.
% -----------------------------------------------------------------------------

\def \topDirectory {../..}

\documentclass[10pt, compress]{beamer}

\usepackage{\topDirectory/template/style/directives}
%%%%%%%%%%%%%%%%%%%%%%%%%%%%%%%%%%%%%%%%%%%%%%%%%%%%%%%%%%%%%%%%%%%%%%%%%%%%%%
% CS114: Introduction to Programming in Java
% Copyright 2015 Pejman Ghorbanzade <mail@ghorbanzade.com>
% Creative Commons Attribution-ShareAlike 4.0 International License
% https://github.com/ghorbanzade/UMB-CS114-2015F/blob/master/LICENSE
%%%%%%%%%%%%%%%%%%%%%%%%%%%%%%%%%%%%%%%%%%%%%%%%%%%%%%%%%%%%%%%%%%%%%%%%%%%%%%

\course{id}{CS114}
\course{name}{Introduction to Java}
\course{venue}{Mon/Wed, 5:30 PM - 6:45 PM}
\course{semester}{Fall 2015}
\course{department}{Department of Computer Science}
\course{university}{University of Massachusetts Boston}

\instructor{name}{Pejman Ghorbanzade}
\instructor{title}{}
\instructor{position}{Student Instructor}
\instructor{email}{pejman@cs.umb.edu}
\instructor{phone}{617-287-6419}
\instructor{office}{S-3-124B}
\instructor{office-hours}{Mon/Wed 16:00-17:30}
\instructor{address}{University of Massachusetts Boston, 100 Morrissey Blvd., Boston, MA}

\usepackage{\topDirectory/template/style/beamerthemeUmassLecture}
\doc{number}{13}
%\setbeamertemplate{footline}[text line]{}

\begin{document}
\prepareCover

\section{Course Administration}

\begin{slide}
	\begin{block}{Overview}
		\begin{itemize}
			\item[] Design Strategy
			\item[] Constructors
		\end{itemize}
	\end{block}
\end{slide}

\section{Design Strategy}

\begin{slide}
	\begin{block}{Objective}
		Write a program \texttt{CatTest.java} that instantiates two cats, named \textit{Kitty} and \textit{Jinxy}, from class \texttt{Cat.java}.

		Assuming the two cats are initially positioned respectively at coordinate $(0, 0)$ and $(2, 4)$, your program should allow user to control movements of the two cats in two dimensional space.

		The program terminates when user enters $(0, 0)$ for movement of a cat.
	\end{block}
\end{slide}

\begin{slide}
	\begin{block}{Sample Run}
		\begin{verbatim}
			$ javac Cat CatTest
			$ java CatTest
			Enter movement of Kitty: 5 5
				Turnip is at (5, 5).
			Enter movement of Jinxy: 1 4
				Jinxy is at (3, 8).
			Enter movement of Kitty: 3 8
				Turnip is at (8, 13).
			Enter movement of Jinxy: 1 4
				Jinxy is at (4, 12).
			Enter movement of Kitty: 0 0
				Turnip is at (8, 13).
		\end{verbatim}
	\end{block}
\end{slide}

\begin{slide}
	\begin{block}{Design Approach}
		\begin{enumerate}
			\item[] Identify the objects of concern.
			\item[] Identify states of objects.
			\item[] Identify behavior of objects.
			\item[] Make blueprints of objects (Classes).
			\item[] Instantiate objects from blueprints.
			\item[] Declare attributes of objects.
			\item[] Use objects to achieve programming goal.
		\end{enumerate}
	\end{block}
\end{slide}

\begin{slide}
	\begin{block}{Design Approach}
		\begin{description}
			\item[Objects of Concern] Kitty, Jinxy.
			\item[States of Objects] Name, Position in X axis, Position in Y axis.
			\item[Behavior of Objects] Move.
			\item[Required Classes] Cat.
		\end{description}
	\end{block}
\end{slide}

\begin{slide}
	\begin{block}{Framework of a Class}
		\begin{minted}[fontsize=\small,tabsize=8, linenos, firstnumber=1]{java}
			public class ClassName {
			    // specify fields (class members)
			    // specify attributes (instance members)
			    // specify static methods
			    // specify instance methods
			    // specify constructors
			    // etc.
			}
		\end{minted}
	\end{block}
\end{slide}

\begin{slide}
	\begin{block}{Data Members}
		Data Members are variables inside Classes in which data is stored.

		Sometimes these variables may represent general properties of the class (fields) in which case they are called \textit{class members}.

		Sometimes they may represent properties of specific objects instantiated from that class (attributes) in which case they are called \textit{instance members}.
	\end{block}
\end{slide}

\begin{slide}
	\begin{block}{Class Members}
		Class members are variables that are common to all objects.
		They are declared with keyword \texttt{static} and thus are sometimes called \emph{static fields}.
		\begin{minted}[fontsize=\small,tabsize=8, linenos, firstnumber=1]{java}
			public class Cat {
			    public static int numberOfCats = 0;
			    // specify attributes (instance members)
			    // specify static methods
			    // specify instance methods
			    // specify constructors
			    // etc.
			}
		\end{minted}
	\end{block}
\end{slide}

\begin{slide}
	\begin{block}{Instance Members}
		Instance variables are variables that define states of an object.
		\begin{minted}[fontsize=\small,tabsize=8, linenos, firstnumber=1]{java}
			public class Cat {
			    public static int numberOfCats = 0;
			    public String name;
			    public double posX;
			    public double posY;
			    // specify static methods
			    // specify instance methods
			    // specify constructors
			    // etc.
			}
		\end{minted}
	\end{block}
\end{slide}

\begin{slide}
	\begin{block}{Static Methods}
		Static methods are methods that belong to the class rather than objects.
		They are often used to access and process class members.

		It is best practice to avoid or minimize the use of static methods.
		\begin{minted}[fontsize=\small,tabsize=8, linenos, firstnumber=1]{java}
			public class Cat {
			    public static int numberOfCats = 0;
			    public String name;
			    public double posX;
			    public double posY;
			    public static int getNumberOfCats() {
			        return numberOfCats;
			    }
			    // specify instance methods
			    // specify constructors
			    // etc.
			}
		\end{minted}
	\end{block}
\end{slide}

\begin{slide}
	\begin{block}{Instance Methods}
		Instance methods are methods defined for objects constructed from the class.
		\begin{minted}[fontsize=\small,tabsize=8, linenos, firstnumber=9]{java}
			    public void move(double posX, double posY) {
			        this.posX += posX;
			        this.posY += posY;
			    }
			    public void showPosition() {
			        System.out.printf("%s is at (%.1f, %.1f).\n",
			            this.name, this.posX, this.posY
			        );
			    }
			    // specify constructors
			    // etc.
			}
		\end{minted}
	\end{block}
\end{slide}

\begin{slide}
	\begin{block}{Design Approach}
		\begin{enumerate}
			\item[] Identify the objects of concern.
			\item[] Identify states of objects.
			\item[] Identify behavior of objects.
			\item[] Make blueprints of objects (Classes).
			\item[] Instantiate objects from blueprints.
			\item[] Declare attributes of objects.
			\item[] Use objects to achieve programming goal.
		\end{enumerate}
	\end{block}
\end{slide}

\begin{slide}
	\begin{block}{\texttt{CatTest.java} (v1.0) (Page 1)}
		\begin{minted}[fontsize=\small,tabsize=8, linenos, firstnumber=1]{java}
			import java.util.Scanner;
			public class CatTest {
			    public static void main(String[] args) {
			        Scanner input = new Scanner();
			        Cat cat1 = new Cat();
			        cat1.name = "Kitty";
			        cat1.posX = 0;
			        cat1.posY = 0;
			        Cat cat2 = new Cat();
			        cat2.name = "Jinxy";
			        cat2.posX = 2;
			        cat2.posY = 4;
		\end{minted}
	\end{block}
\end{slide}

\begin{slide}
	\begin{block}{\texttt{CatTest.java} (v1.0) (Page 2)}
		\begin{minted}[fontsize=\small,tabsize=8, linenos, firstnumber=13]{java}
			        int[] movements;
			        while (true) {
			            movements = promptMovements();
			            cat1.move(movements[0], movements[1]);
			            cat1.showPosition();
			            if (movements[0] == 0 && movements[1] == 0)
			                break;
			            movements = promptMovements();
			            cat2.move(movements[0], movements[1]);
			            cat2.showPosition();
			            if (movements[0] == 0 && movements[1] == 0)
			                break;
			        }
			    }
			}
		\end{minted}
	\end{block}
\end{slide}

\section{Constructors}

\begin{slide}
	\begin{block}{Problem Statement}
		Upon instantiation, states of the object are not initialized.
		\begin{enumerate}
			\item
			\begin{minted}[fontsize=\small,tabsize=8]{java}
				Cat cat1 = new Cat();
				cat1.showPosition();
			\end{minted}
			\item
			\begin{minted}[fontsize=\small,tabsize=8]{java}
				Cat cat2 = new Cat();
				System.out.println(cat2.name);
			\end{minted}
		\end{enumerate}
	\end{block}
\end{slide}

\begin{slide}
	\begin{block}{How do we think?}
		\begin{quote}
		Aliens visited the International Space Station last night and drank a cup of tea with Scott Kelly.
		\end{quote}
	\end{block}
	\pause
	\begin{block}{Problem Statement}
		Initializing states of an object after instantiation is inconsistent with how we think.
	\end{block}
\end{slide}

\begin{slide}
	\begin{block}{Definition}
		A constructor is a block of code that is automatically invoked upon instantiation.
		\begin{minted}[fontsize=\small, tabsize=8]{java}
		Cat mycat = new Cat();
		\end{minted}
		\begin{description}
			\item[Cat] is a reference data type.
			\item[mycat] is the identifier for our object.
			\item[new] is the keyword to instantiate an object from class \texttt{Cat}, allocating memory for the object.
			\item[Cat()] is a constructor to initialize states of the object.
		\end{description}
	\end{block}
\end{slide}

\begin{slide}
	\begin{block}{Structure}
		\begin{itemize}
			\item[] modifier
			\item[] \emph{parameter list}
			\item[] constructor body
		\end{itemize}
	\end{block}
	\begin{block}{But, wait...}
		Did we define a constructor for \texttt{Cat.java}?

		If no constructor is declared in a class, JVM will append a default no-argument constructor of the following form and will invoke it each time an object is instantiated from that class.
		\begin{minted}[fontsize=\small, tabsize=8]{java}
			public Cat() {
			}
		\end{minted}
	\end{block}
\end{slide}

\begin{slide}
	\begin{block}{\texttt{Cat.java}}
		\begin{minted}[fontsize=\small,tabsize=8, linenos, firstnumber=18]{java}
			public Cat(String name, double posX, double posY) {
			    this.name = name;
			    this.posX = posX;
			    this.posY = posY;
			    numberOfCats++;
			}
		\end{minted}
	\end{block}
\end{slide}

\begin{slide}
	\begin{block}{\texttt{CatTest.java} (v2.0) (Page 1)}
		\begin{minted}[fontsize=\small,tabsize=8, linenos, firstnumber=1]{java}
			import java.util.Scanner;
			public class CatTest {
			    public static void main(String[] args) {
			        Scanner input = new Scanner();
			        Cat cat1 = new Cat("Kitty", 0, 0);
			        Cat cat2 = new Cat("Jinxy", 2, 4);
		\end{minted}
	\end{block}
\end{slide}

\begin{slide}
	\begin{block}{\texttt{CatTest.java} (v2.0) (Page 2)}
		\begin{minted}[fontsize=\small,tabsize=8, linenos, firstnumber=7]{java}
			        int[] movements;
			        while (true) {
			            movements = promptMovements();
			            cat1.move(movements[0], movements[1]);
			            cat1.showPosition();
			            if (movements[0] == 0 && movements[1] == 0)
			                break;
			            movements = promptMovements();
			            cat2.move(movements[0], movements[1]);
			            cat2.showPosition();
			            if (movements[0] == 0 && movements[1] == 0)
			                break;
			        }
			    }
			}
		\end{minted}
	\end{block}
\end{slide}

\begin{slide}
	\begin{block}{Remember}
		Java can distinguish between constructors with different signatures.
		A constructor does not have a return type.
		Signature of a constructor is defined by its parameter list.
		\begin{minted}[fontsize=\small,tabsize=8, linenos, firstnumber=18]{java}
			public Cat(String name, double[] position) {
			    this.name = name;
			    this.posX = position[0];
			    this.posY = position[1];
			    numberOfCats++;
			}
			public Cat(String name, double posX, double posY) {
			    this.name = name;
			    this.posX = posX;
			    this.posY = posY;
			    numberOfCats++;
			}
		\end{minted}
	\end{block}
\end{slide}

\plain{}{Keep Calm\\and\\Practice}

\end{document}
