% -----------------------------------------------------------------------------
% The MIT License (MIT)
%
% Copyright (c) 2015 Pejman Ghorbanzade
%
% Permission is hereby granted, free of charge, to any person obtaining a copy
% of this software and associated documentation files (the "Software"), to deal
% in the Software without restriction, including without limitation the rights
% to use, copy, modify, merge, publish, distribute, sublicense, and/or sell
% copies of the Software, and to permit persons to whom the Software is
% furnished to do so, subject to the following conditions:
%
% The above copyright notice and this permission notice shall be included in
% all copies or substantial portions of the Software.
%
% THE SOFTWARE IS PROVIDED "AS IS", WITHOUT WARRANTY OF ANY KIND, EXPRESS OR
% IMPLIED, INCLUDING BUT NOT LIMITED TO THE WARRANTIES OF MERCHANTABILITY,
% FITNESS FOR A PARTICULAR PURPOSE AND NONINFRINGEMENT. IN NO EVENT SHALL THE
% AUTHORS OR COPYRIGHT HOLDERS BE LIABLE FOR ANY CLAIM, DAMAGES OR OTHER
% LIABILITY, WHETHER IN AN ACTION OF CONTRACT, TORT OR OTHERWISE, ARISING FROM,
% OUT OF OR IN CONNECTION WITH THE SOFTWARE OR THE USE OR OTHER DEALINGS IN
% THE SOFTWARE.
% -----------------------------------------------------------------------------

\def \topDirectory {../..}

\documentclass[10pt, compress]{beamer}

\usepackage{\topDirectory/template/style/directives}
%%%%%%%%%%%%%%%%%%%%%%%%%%%%%%%%%%%%%%%%%%%%%%%%%%%%%%%%%%%%%%%%%%%%%%%%%%%%%%
% CS114: Introduction to Programming in Java
% Copyright 2015 Pejman Ghorbanzade <mail@ghorbanzade.com>
% Creative Commons Attribution-ShareAlike 4.0 International License
% https://github.com/ghorbanzade/UMB-CS114-2015F/blob/master/LICENSE
%%%%%%%%%%%%%%%%%%%%%%%%%%%%%%%%%%%%%%%%%%%%%%%%%%%%%%%%%%%%%%%%%%%%%%%%%%%%%%

\course{id}{CS114}
\course{name}{Introduction to Java}
\course{venue}{Mon/Wed, 5:30 PM - 6:45 PM}
\course{semester}{Fall 2015}
\course{department}{Department of Computer Science}
\course{university}{University of Massachusetts Boston}

\instructor{name}{Pejman Ghorbanzade}
\instructor{title}{}
\instructor{position}{Student Instructor}
\instructor{email}{pejman@cs.umb.edu}
\instructor{phone}{617-287-6419}
\instructor{office}{S-3-124B}
\instructor{office-hours}{Mon/Wed 16:00-17:30}
\instructor{address}{University of Massachusetts Boston, 100 Morrissey Blvd., Boston, MA}

\usepackage{\topDirectory/template/style/beamerthemeUmassLecture}
\doc{number}{10}
%\setbeamertemplate{footline}[text line]{}

\begin{document}
\prepareCover

\section{Course Administration}

\begin{slide}
	\begin{itemize}
		\item[] Assignment 2 due. Solution released online.
		\item[] Assignment 3 released. Due on November 2, 2015 at 5:30 PM.
		\item[] Midterm exam to be held Ocober 21, 2015 at 5:30 PM.
	\end{itemize}
\end{slide}

\begin{slide}
	\begin{block}{Overview}
		\begin{itemize}
			\item[] Multi-Dimensional Arrays
		\end{itemize}
	\end{block}
\end{slide}

\section{Multi-Dimensional Arrays}

\begin{slide}
	\begin{block}{Dimension Definition}
		The dimension of an array is the number of indices needed to select an element.
	\end{block}
	\begin{block}{Initialization}
		\begin{minted}[fontsize=\small,tabsize=8]{java}
			int[][] matrix = new int[3][3];
			for (int i = 0; i < 3; i++)
			    for (int j = 0; j < 3; j++)
			        matrix[i][j] = 0;
		\end{minted}
	\end{block}
\end{slide}

\begin{slide}
	\begin{block}{Objective}
		Write a program \texttt{MagicMatrix.java} that takes number $N$ and generates a magic matrix of size $N$ using numbers $1$ to $N^2$.
	\end{block}
\end{slide}

\begin{slide}
	\begin{block}{Simplification}
		\begin{itemize}
			\item[] Get matrix size N from user
			\item[] Build an array of length $N^2$ from 1 to N
			\item[] Shuffle the array
			\item[] Convert 1-D Array to 2-D Array
			\item[] Print elements of array
		\end{itemize}
	\end{block}
\end{slide}

\begin{slide}
	\begin{block}{\texttt{MagicMatrix.java} (part 1)}
		\begin{minted}[fontsize=\small,tabsize=8, linenos, firstnumber=1]{java}
			public class Array {
			    public static void main(String[] args) {
			        // get matrix size
			        Scanner input = new Scanner(System.in);
			        int num = input.nextInt();
			        input.close();
			        // initialize array
			        int[] numbers = new int[num * num];
			        for (int i = 0; i < numbers.length; i++)
			            numbers[i] = i+1;
		\end{minted}
	\end{block}
\end{slide}

\begin{slide}
	\begin{block}{\texttt{MagicMatrix.java} (part 2)}
		\begin{minted}[fontsize=\small,tabsize=8, linenos, firstnumber=12]{java}
			// shuffle array
			for (int i = 0; i < numbers.length; i++) {
			    int j = (int) (i + (numbers.length - i) * Math.random());
			    int temp = numbers[j];
			    numbers[j] = numbers[i];
			    numbers[i] = temp;
			}
		\end{minted}
	\end{block}
\end{slide}

\begin{slide}
	\begin{block}{\texttt{MagicMatrix.java} (part 3)}
		\begin{minted}[fontsize=\small,tabsize=8, linenos, firstnumber=19]{java}
			        // convert shuffled array to matrix
			        int[][] matrix = new int[num][num];
			        for (int i = 0; i < numbers.length; i++) {
			            matrix[i/num][i%num] = numbers[i];
			        }
			        // show matrix
			        for (int i = 0; i < num; i++) {
			            for (int j = 0; j < num; j++)
			                System.out.print(matrix[i][j]);
			            System.out.println();
			        }
			    }
			}
		\end{minted}
	\end{block}
\end{slide}

\begin{slide}
	\begin{block}{Higher Dimensional Arrays}
		\begin{minted}[fontsize=\small,tabsize=8, linenos, firstnumber=1]{java}
			int num = 1;
			int[][][] cubicMatrix = new int[3][3][3];
			for (int i = 1; i < 3; i++)
			    for (int j = 1; j < 3; j++)
			        for (int k = 1; k < 3; k++)
			            cubicMatrix[i][j][k] = num++;
		\end{minted}
	\end{block}
\end{slide}

\begin{slide}
	\begin{block}{Remember}
		\begin{itemize}
			\item[] Arrays are simple data structures for data organization
			\item[] Arrays provide an efficient way to access data
			\item[] Elements of array are of same type
			\item[] Arrays have fixed-size
		\end{itemize}
	\end{block}
\end{slide}

\plain{}{Keep Calm\\and\\Practice}

\end{document}
