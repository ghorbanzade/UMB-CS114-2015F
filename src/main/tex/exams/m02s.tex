%%%%%%%%%%%%%%%%%%%%%%%%%%%%%%%%%%%%%%%%%%%%%%%%%%%%%%%%%%%%%%%%%%%%%%
% UMB-CS114-2015F: Introduction to Programming in Java
% Copyright 2015 Pejman Ghorbanzade <pejman@ghorbanzade.com>
% Creative Commons Attribution-ShareAlike 4.0 International License
% More info: https://github.com/ghorbanzade/UMB-CS114-2015F
%%%%%%%%%%%%%%%%%%%%%%%%%%%%%%%%%%%%%%%%%%%%%%%%%%%%%%%%%%%%%%%%%%%%%%

\def \topDirectory {.}
\def \texDirectory {\topDirectory/src/main/tex}

\documentclass[12pt,letterpaper,twoside]{article}
\usepackage{\texDirectory/template/style/directives}
\usepackage{\texDirectory/template/style/assignment}
%%%%%%%%%%%%%%%%%%%%%%%%%%%%%%%%%%%%%%%%%%%%%%%%%%%%%%%%%%%%%%%%%%%%%%%%%%%%%%
% CS114: Introduction to Programming in Java
% Copyright 2015 Pejman Ghorbanzade <mail@ghorbanzade.com>
% Creative Commons Attribution-ShareAlike 4.0 International License
% https://github.com/ghorbanzade/UMB-CS114-2015F/blob/master/LICENSE
%%%%%%%%%%%%%%%%%%%%%%%%%%%%%%%%%%%%%%%%%%%%%%%%%%%%%%%%%%%%%%%%%%%%%%%%%%%%%%

\course{id}{CS114}
\course{name}{Introduction to Java}
\course{venue}{Mon/Wed, 5:30 PM - 6:45 PM}
\course{semester}{Fall 2015}
\course{department}{Department of Computer Science}
\course{university}{University of Massachusetts Boston}

\instructor{name}{Pejman Ghorbanzade}
\instructor{title}{}
\instructor{position}{Student Instructor}
\instructor{email}{pejman@cs.umb.edu}
\instructor{phone}{617-287-6419}
\instructor{office}{S-3-124B}
\instructor{office-hours}{Mon/Wed 16:00-17:30}
\instructor{address}{University of Massachusetts Boston, 100 Morrissey Blvd., Boston, MA}


\usepackage{amsmath}

\begin{document}

\doc{title}{Solution to Midterm Exam}
\doc{points}{20}

\prepare{header}

\section*{Question 1}

The following programs have each a \textbf{single} compilation error given below.
You are expected to find and fix the error so that given command-line arguments will lead to expected output as indicated.

\begin{enumerate}[label=\textbf{(\alph*)}]

\item Execution Script \hfill Expected Output\\
\texttt{java HelloWorld President} \hfill \texttt{Hello President!}
\begin{lstlisting}
public class HelloWorld {
	public static void main(String[] args) {
		System.out.println("Hello + args[0] + "!");
	}
}
\end{lstlisting}

\begin{terminal}
HelloWorld.java:3: error: ')' expected
    System.out.println("Hello + args[0] + "!");
HelloWorld.java:3: error: unclosed string literal
    System.out.println("Hello + args[0] + "!");
HelloWorld.java:3: error: ';' expected
    System.out.println("Hello + args[0] + "!");
HelloWorld.java:5: error: reached end of file while parsing
}
4 errors
\end{terminal}

\newpage

\item Execution Script \hfill Expected Output\\
\texttt{java Feet2Centimeters 5} \hfill \texttt{152.4}
\begin{lstlisting}
public class Feet2Centimeters {
	public static void main(String[] args) {
		double conversion = 30.48;
		double cm = args[0] * conversion;
		System.out.println(cm);
	}
}
\end{lstlisting}

\begin{terminal}
Feet2Centimeters.java:4: error: bad operand types for binary operator '*'
        double cm = args[0] * conversion;
1 error
\end{terminal}

\item Execution Script \hfill Expected Output\\
\texttt{java SphereSurface 7} \hfill \texttt{615.75}

\begin{lstlisting}
public class SphereSurface {
	public static void main(String[] args) {
		double radius = Double.parseDouble(args[0]);
		double surface = 4 * Math.PI * pow(radius, 2);
		System.out.printf("%.2f\n", surface);
	}
}
\end{lstlisting}

\begin{terminal}
SphereSurface.java:4: error: cannot find symbol
        double surface = 4 * Math.PI * pow(radius, 2);
1 error
\end{terminal}

\newpage

\item Execution Script \hfill Expected Output\\
\texttt{java Factorial 5} \hfill \texttt{120}

\begin{lstlisting}
public class Factorial {
	public static void main(String[] args) {
		int num = Integer.parseInt(args[0]);
		int counter = 1;
		int result = 1;
		do
			result *= counter;
			counter++;
		while (counter <= num);
		System.out.println(result);
	}
}
\end{lstlisting}

\begin{terminal}
Factorial.java:7: error: while expected
	result *= counter;
Factorial.java:8: error: illegal start of expression
	counter++;
Factorial.java:8: error: ';' expected
	counter++;
Factorial.java:9: error: not a statement
	while (counter <= num);
4 errors
\end{terminal}

\end{enumerate}

\newpage

\subsection*{Solution}

\begin{enumerate}[label=\textbf{(\alph*)}]
\item Execution Script \hfill Expected Output\\
\texttt{java HelloWorld President} \hfill \texttt{Hello President!}

\lstset{language=Java,tabsize=4}
\begin{lstlisting}
public class HelloWorld {
	public static void main(String[] args) {
		System.out.println("Hello" + args[0] + "!");
	}
}
\end{lstlisting}

\item Execution Script \hfill Expected Output\\
\texttt{java Feet2Centimeters 5} \hfill \texttt{152.4}

\lstset{language=Java,tabsize=4}
\begin{lstlisting}
public class Feet2Centimeters {
	public static void main(String[] args) {
		double conversion = 30.48;
		double cm = Double.parseDouble(args[0]) * conversion;
		System.out.println(cm);
	}
}
\end{lstlisting}

\item Execution Script \hfill Expected Output\\
\texttt{java SphereSurface 7} \hfill \texttt{615.75}

\lstset{language=Java,tabsize=4}
\begin{lstlisting}
public class SphereSurface {
	public static void main(String[] args) {
		double radius = Double.parseDouble(args[0]);
		double surface = 4 * Math.PI * Math.pow(radius, 2);
		System.out.printf("%.2f\n", surface);
	}
}
\end{lstlisting}

\item Execution Script \hfill Expected Output\\
\texttt{java Factorial 5} \hfill \texttt{120}

\lstset{language=Java,tabsize=4}
\begin{lstlisting}
public class Factorial {
	public static void main(String[] args) {
		int num = Integer.parseInt(args[0]);
		int counter = 1;
		int result = 1;
		do {
			result *= counter;
			counter++;
		}
		while (counter <= num);
		System.out.println(result);
	}
}
\end{lstlisting}

\end{enumerate}

\section*{Question 2}
The following code snippet had been written on a scrap of paper found in a trashcan of a classroom.
The program is error-free and functions as expected but unfortunately no one knows what it does.
Read the code snippet and explain how it works.
No assumption can be made about user input except that the input is always an integer.

\begin{lstlisting}
import java.util.Scanner;
public class FiveStar {
	public static void main(String[] args) {
		Scanner key = new Scanner(System.in);
		System.out.printf("Insert Number: ");
		int num = key.nextInt();
		int rand = (int) (Math.random() * 100) + 1;
		double result = num * Math.pow(1, rand);
		double sum = 0;
		int var = 0;
		while (var != 6) {
			sum += Math.pow(-1, var) * result;
			var++;
		}
		System.out.printf("Result is %.2f\n", sum);
	}
}
\end{lstlisting}

\subsection*{Solution}

The program prompts user for an integer number $x$.
A random integer number $y$ in range 0 to 100 is generated and is used to calculate the value $z$ to be stored in variable \texttt{result}.
Since $z = x \times 1^{y} = x$, the value stored in \texttt{result} is always the number given by the user.
The program continues in a \texttt{while} loop with 6 iterations computing variable \texttt{sum} according to Equation \ref{eq1} given below.

\begin{equation}
\text{sum} = \sum\limits_{i=0}^5 (-1)^i \times \text{result} = 0
\label{eq1}
\end{equation}

Since the final value stored in \texttt{sum} is $0$, regardless of the use input, what is printed by the program would always be zero.
Following is a sample run of the program.

\begin{terminal}
$ java FiveStar.java
$ java FiveStar
Insert Number: 5
Result is 0.00
\end{terminal}

\section*{Question 3}

Write a program \texttt{NumberLimit.java} that prompts user for a strictly positive integer number $x$ and prints the largest non-negative integer number $y$ such that $y^3 < x$.
For simplicity, assume the user input is always a strictly positive integer number.
Following is an expected sample run of your program.

\begin{terminal}
$ javac NumberLimit.java
$ java NumberLimit
Upper Limit? 70
Number is 4.
\end{terminal}

\subsection*{Solution}

\lstset{language=Java,tabsize=4}
\begin{lstlisting}
import java.util.Scanner;
public class NumberLimit {
	public static void main(String[] args) {
		Scanner input = new Scanner(System.in);
		System.out.print("Upper Limit? ");
		int num = input.nextInt();
		input.close();
		int i = 0;
		while (Math.pow(i, 3) < num)
			i++;
		System.out.printf("Number is %d.\n", i - 1);
	}
}
\end{lstlisting}

\end{document}
