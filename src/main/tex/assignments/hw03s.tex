%%%%%%%%%%%%%%%%%%%%%%%%%%%%%%%%%%%%%%%%%%%%%%%%%%%%%%%%%%%%%%%%%%%%%%
% UMB-CS114-2015F: Introduction to Programming in Java
% Copyright 2015 Pejman Ghorbanzade <pejman@ghorbanzade.com>
% Creative Commons Attribution-ShareAlike 4.0 International License
% More info: https://github.com/ghorbanzade/UMB-CS114-2015F
%%%%%%%%%%%%%%%%%%%%%%%%%%%%%%%%%%%%%%%%%%%%%%%%%%%%%%%%%%%%%%%%%%%%%%

\def \topDirectory {.}
\def \texDirectory {\topDirectory/src/main/tex}

\documentclass[12pt,letterpaper,twoside]{article}
\usepackage{\texDirectory/template/style/directives}
\usepackage{\texDirectory/template/style/assignment}
%%%%%%%%%%%%%%%%%%%%%%%%%%%%%%%%%%%%%%%%%%%%%%%%%%%%%%%%%%%%%%%%%%%%%%%%%%%%%%
% CS114: Introduction to Programming in Java
% Copyright 2015 Pejman Ghorbanzade <mail@ghorbanzade.com>
% Creative Commons Attribution-ShareAlike 4.0 International License
% https://github.com/ghorbanzade/UMB-CS114-2015F/blob/master/LICENSE
%%%%%%%%%%%%%%%%%%%%%%%%%%%%%%%%%%%%%%%%%%%%%%%%%%%%%%%%%%%%%%%%%%%%%%%%%%%%%%

\course{id}{CS114}
\course{name}{Introduction to Java}
\course{venue}{Mon/Wed, 5:30 PM - 6:45 PM}
\course{semester}{Fall 2015}
\course{department}{Department of Computer Science}
\course{university}{University of Massachusetts Boston}

\instructor{name}{Pejman Ghorbanzade}
\instructor{title}{}
\instructor{position}{Student Instructor}
\instructor{email}{pejman@cs.umb.edu}
\instructor{phone}{617-287-6419}
\instructor{office}{S-3-124B}
\instructor{office-hours}{Mon/Wed 16:00-17:30}
\instructor{address}{University of Massachusetts Boston, 100 Morrissey Blvd., Boston, MA}


\usepackage{amsmath}

\begin{document}

\doc{title}{Solution to Assignment 3}
\doc{date-pub}{Oct 19, 2015 at 5:30 PM}
\doc{date-due}{Nov 04, 2015 at 5:30 PM}
\doc{points}{8}

\prepare{header}

\section*{Question 1}

Write a program \texttt{HouseOfCards.java} that randomly selects two cards from a standard 52-card deck.
The program would then print the names of the cards based on their rank and suit.
Finally, your program should determine if the two cards share the same rank or the same suit.
Your program is expected to function as shown in following examples:

\begin{terminal}
$ java HouseOfCards
Queen of Diamonds
Jack of Spades
Cards do not share rank or suit.
$ java HouseOfCards
Ace of Hearts
8 of Hearts
Cards share the same suit.
\end{terminal}

\subsection*{Solution}

\lstset{language=Java,tabsize=4}
\begin{lstlisting}
public class HouseOfCards {
	public static void main(String[] args) {
		int[] card1 = generateCard();
		int[] card2 = generateCard();
		printCardName(card1[0], card1[1]);
		printCardName(card2[0], card2[1]);
		if (card1.equals(card2))
			System.out.println("Cards are the same");
		else if (card1[0] == card2[0])
			System.out.println("Cards share the same suit.");
		else if (card1[1] == card2[1])
			System.out.println("Cards share the same rank.");
		else
			System.out.println("Cards do not share rank or suit.");
	}
	public static int[] generateCard() {
		int[] card = new int[2];
		int num = (int) Math.floor(Math.random() * 52);
		card[0] = num / 13;
		card[1] = num % 13;
		return card;
	}
	public static void printCardName(int suitNum, int rankNum) {
		String[] suitNames = {
			"Spades", "Diamonds", "Clubs", "Hearts"
		};
		String[] cardNames = {
			"Ace", "Two", "Three", "Four", "Five",
			"Six", "Seven", "Eight", "Nine", "Ten",
			"Soldier", "Queen", "King"
		};
		System.out.printf("%s of %s\n", cardNames[rankNum], suitNames[suitNum]);
	}
}
\end{lstlisting}

\section*{Question 2}

Write a program \texttt{Deviation.java} that prompts user to enter a positive integer number $n$ and then asks user to enter $n$ floating-point numbers.
Your program should compute and display standard deviation of the numbers according to Equation \ref{eq1}, in the format shown below.

\begin{equation}
\text{deviation} = \sqrt{\frac{\sum_{i = 1}^{n} \left( x_i - \frac{\sum_{i = 1}^{n} x_i}{n} \right)^2 }{n - 1}}
\label{eq1}
\end{equation}

\begin{terminal}
$ javac Deviation.java
$ java Deviation
Enter count: 10
Enter 10 numbers: 1.9 2.5 3.7 2 1 6 3 4 5 2
Standard Deviation: 1.557
\end{terminal}

\newpage

\subsection*{Solution}

\lstset{language=Java,tabsize=4}
\begin{lstlisting}
import java.util.Scanner;
public class Deviation {
	public static void main(String[] args) {
		double[] numbers = getNumbers();
		double deviation = findDeviation(numbers);
		System.out.printf("Standard Deviation: %.3f\n", deviation);
	}
	public static double[] getNumbers() {
		Scanner input = new Scanner(System.in);
		System.out.printf("Enter count: ");
		int num = input.nextInt();
		double[] numbers = new double[num];
		System.out.printf("Enter %d Numbers: ", num);
		for (int i = 0; i < num; i++) {
			numbers[i] = input.nextDouble();
		}
		input.close();
		return numbers;
	}
	public static double findDeviation(double[] numbers) {
		double mean = findMean(numbers);
		double sum = 0;
		for (int i = 0; i < numbers.length; i++)
			sum += Math.pow(numbers[i] - mean, 2);
		double deviation = Math.sqrt(sum / (numbers.length - 1));
		return deviation;
	}
	public static double findMean(double[] numbers) {
		double sum = 0;
		for (int i = 0; i < numbers.length; i++)
			sum += numbers[i];
		double mean = sum / numbers.length;
		return mean;
	}
}
\end{lstlisting}

\newpage

\section*{Question 3}

Write a program \texttt{DuplicateTester.java} that prompts user for a positive integer $x$ and creates an array with $x$ randomly generted integer elements in range 0 to 9.
Use a method with signature \texttt{public static int[] removeDuplicates(int[] array);} to remove all duplicate elements in the array.
Your program is expected to terminate by displaying the new array.
Following is a sample expected run of the program.

\begin{terminal}
$ javac DuplicateTester.java
$ java DuplicateTester
Size of Array: 10
Generated Array:
3 4 2 8 2 4 3 1 6 7
Filtered Array:
3 4 2 8 1 6 7
\end{terminal}

\subsection*{Solution}

\lstset{language=Java,tabsize=4}
\begin{lstlisting}
import java.util.Scanner;
public class DuplicateTester {
	public static void main(String[] args) {
		int size = getSize();
		System.out.println("Generated Array:");
		int[] array = generateArray(size);
		display(array);
		System.out.println("Filtered Array:");
		int[] newArray = removeDuplicates(array);
		display(newArray);
	}
	public static int getSize() {
		Scanner input = new Scanner(System.in);
		System.out.print("Size of Array: ");
		int size = input.nextInt();
		input.close();
		return size;
	}
	public static int[] generateArray(int size) {
		int[] array = new int[size];
		for (int i = 0; i < size; i++)
			array[i] = (int) Math.floor(Math.random() * 10);
		return array;
	}
	public static int[] removeDuplicates(int[] array) {
		boolean[] flags = new boolean[10];
		int[] numbers = new int[10];
		for (int i = 0; i < flags.length; i++)
			flags[i] = false;
		int j = 0;
		for (int i = 0; i < array.length; i++) {
			if (flags[array[i]] == false) {
				flags[array[i]] = true;
				numbers[j++] = array[i];
			}
		}
		int[] newArray = new int[j];
		for (int i = 0; i < j; i++)
			newArray[i] = numbers[i];
		return newArray;
	}
	public static void display(int[] array) {
		int size = array.length;
		for (int i = 0; i < size; i++)
			System.out.printf("%d ", array[i]);
		System.out.println();
	}
}
\end{lstlisting}

\section*{Question 4}

Thomas, a CS114 student, had been asked to write a program that performs simple arithmetic operations (add and multiply) on two matrices $x$ and $y$ whose elements are randomly generated from range 0 to 9.
The program was supposed to prompt user for a number $r$ and generate two $r \times r$ matrices, $x$ and $y$.
The program would then display $x + y$ and $x \times y$ respectively, as shown in Figure 1.

\newpage

\renewcommand{\lstlistingname}{Figure}
\lstset{caption=Expected Output of \texttt{MatrixWorld.java}, label=terminal1}
\begin{terminal}
Size of Matrices: 3

Matrix A:
  0   3   2
  7   0   1
  1   6   5

Matrix B:
  7   8   9
  0   3   1
  8   9   6

Sum:
  7  11  11
  7   3   2
  9  15  11

Multiplication:
 16  27  15
 57  65  69
 47  71  45
\end{terminal}

Unfortunately, Thomas gave up while developing the requested program.
He has asked that you continue development of his program without modifying the code he has written.

\lstset{caption=}
\lstset{language=Java,tabsize=4}
\begin{lstlisting}
// written by: Thomas A. Anderson
// <thomas.anderson001@umb.edu>

import java.util.Scanner;
public class MatrixWorld {
	public static void main(String[] args) {
		int size = getSize();
		int[][] matrixA = matrixInit(size);
		int[][] matrixB = matrixInit(size);
		int[][] sum = matrixAdd(matrixA, matrixB);
		int[][] multiplication = matrixMultiply(matrixA, matrixB);
		matrixDisplay(matrixA, "Matrix A");
		matrixDisplay(matrixB, "Matrix B");
		matrixDisplay(sum, "Sum");
		matrixDisplay(multiplication, "Multiplication");
	}
	public static int getSize() {
		// To Be Developed
	}
	public static int[][] matrixInit(int size) {
		// To Be Developed
	}
	public static int[][] matrixAdd(int[][] matrixA, int[][] matrixB) {
		// To Be Developed
	}
	public static int[][] matrixMultiply(int[][] matrixA, int[][] matrixB) {
		// To Be Developed
	}
	public static void matrixDisplay(int[][] matrix, String name) {
		// To Be Developed
	}
}
\end{lstlisting}

\subsection*{Solution}

\lstset{language=Java,tabsize=4}
\begin{lstlisting}
// written by: Thomas A. Anderson
// <thomas.anderson001@umb.edu>

import java.util.Scanner;
public class MatrixWorld {
	public static void main(String[] args) {
		int size = getSize();
		int[][] matrixA = matrixInit(size);
		int[][] matrixB = matrixInit(size);
		int[][] sum = matrixAdd(matrixA, matrixB);
		int[][] multiplication = matrixMultiply(matrixA, matrixB);
		matrixDisplay(matrixA, "Matrix A");
		matrixDisplay(matrixB, "Matrix B");
		matrixDisplay(sum, "Sum");
		matrixDisplay(multiplication, "Multiplication");
	}
	public static int getSize() {
		Scanner input = new Scanner(System.in);
		System.out.print("Size of Matrices: ");
		int size = input.nextInt();
		input.close();
		return size;
	}
	public static int[][] matrixInit(int size) {
		int[][] matrix = new int[size][size];
		for (int i = 0; i < size; i++)
			for (int j = 0; j < size; j++)
				matrix[i][j] = (int) Math.floor(Math.random() * 10);
		return matrix;
	}
	public static int[][] matrixAdd(int[][] matrixA, int[][] matrixB) {
		int size = matrixA.length;
		int[][] sum = new int[size][size];
		for (int i = 0; i < size; i++)
			for (int j = 0; j < size; j++)
				sum[i][j] = matrixA[i][j] + matrixB[i][j];
		return sum;
	}
	public static int[][] matrixMultiply(int[][] matrixA, int[][] matrixB) {
		int size = matrixA.length;
		int[][] multiplication = new int[size][size];
		for (int i = 0; i < size; i++) {
			for (int j = 0; j < size; j++) {
				multiplication[i][j] = 0;
				for (int k = 0; k < size; k++) {
					multiplication[i][j] += matrixA[i][k] * matrixB[k][j];
				}
			}
		}
		return multiplication;
	}
	public static void matrixDisplay(int[][] matrix, String name) {
		int size = matrix.length;
		System.out.printf("\n%s:\n", name);
		for (int i = 0; i < size; i++) {
			for (int j = 0; j < size; j++)
				System.out.printf("%3d ", matrix[i][j]);
			System.out.println();
		}
	}
}
\end{lstlisting}

\newpage

\section*{Question 5}

Write a program \texttt{BackwardPrimes.java} that prompts user for a positive integer number $x$ and displays $x$ prime numbers whose reversal is also a prime.
Numbers 13, 31 and 157 are instances of such numbers.
Following is a sample run of your program.

\begin{terminal}
$ javac BackwardPrimes.java
$ java BackwardPrimes
Number of Primes? 15
13 17 31 37 71 73 79 97 107 113 149 157 167 179 199
\end{terminal}

\subsection*{Solution}

\lstset{language=Java,tabsize=4}
\begin{lstlisting}
import java.util.Scanner;
public class BackwardPrimes {
	public static void main(String[] args) {
		int i = 0;
		int num = 2;
		int count = getNum();
		while (i < count)
			if (checkNumber(num++) == true)
				i++;
		System.out.println();
	}
	public static int getNum() {
		Scanner input = new Scanner(System.in);
		System.out.print("Number of Primes: ");
		int num = input.nextInt();
		input.close();
		return num;
	}
	public static boolean checkNumber(int number) {
		boolean flag = false;
		int reversed = reverse(number);
		if (number != reversed) {
			if (isPrime(number) && isPrime(reversed)) {
				System.out.printf("%d ", number);
				flag = true;
			}
		}
		return flag;
	}
	public static boolean isPrime(int number) {
		boolean isPrime = true;
		for (int i = 2; i < number; i++) {
			if (number % i == 0) {
				isPrime = false;
				break;
			}
		}
		return isPrime;
	}
	public static int reverse(int number) {
		String str = Integer.toString(number);
		String strReversed = reverse(str);
		int numReversed = Integer.parseInt(strReversed);
		return numReversed;
	}
	public static String reverse(String str) {
		char[] strReversed = new char[str.length()];
		for (int i = 0; i < str.length(); i++)
			strReversed[i] = str.charAt(str.length() - i - 1);
		return new String(strReversed);
	}
}
\end{lstlisting}

\end{document}
