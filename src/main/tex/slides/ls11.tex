%%%%%%%%%%%%%%%%%%%%%%%%%%%%%%%%%%%%%%%%%%%%%%%%%%%%%%%%%%%%%%%%%%%%%%%%%%%%%%
% CS114: Introduction to Programming in Java
% Copyright 2015 Pejman Ghorbanzade <mail@ghorbanzade.com>
% Creative Commons Attribution-ShareAlike 4.0 International License
% https://github.com/ghorbanzade/UMB-CS114-2015F/blob/master/LICENSE
%%%%%%%%%%%%%%%%%%%%%%%%%%%%%%%%%%%%%%%%%%%%%%%%%%%%%%%%%%%%%%%%%%%%%%%%%%%%%%

\def \topDirectory {../..}
\def \texDirectory {\topDirectory/src/main/tex}

\documentclass[10pt, compress]{beamer}

\usepackage{\texDirectory/template/style/directives}
%%%%%%%%%%%%%%%%%%%%%%%%%%%%%%%%%%%%%%%%%%%%%%%%%%%%%%%%%%%%%%%%%%%%%%%%%%%%%%
% CS114: Introduction to Programming in Java
% Copyright 2015 Pejman Ghorbanzade <mail@ghorbanzade.com>
% Creative Commons Attribution-ShareAlike 4.0 International License
% https://github.com/ghorbanzade/UMB-CS114-2015F/blob/master/LICENSE
%%%%%%%%%%%%%%%%%%%%%%%%%%%%%%%%%%%%%%%%%%%%%%%%%%%%%%%%%%%%%%%%%%%%%%%%%%%%%%

\course{id}{CS114}
\course{name}{Introduction to Java}
\course{venue}{Mon/Wed, 5:30 PM - 6:45 PM}
\course{semester}{Fall 2015}
\course{department}{Department of Computer Science}
\course{university}{University of Massachusetts Boston}

\instructor{name}{Pejman Ghorbanzade}
\instructor{title}{}
\instructor{position}{Student Instructor}
\instructor{email}{pejman@cs.umb.edu}
\instructor{phone}{617-287-6419}
\instructor{office}{S-3-124B}
\instructor{office-hours}{Mon/Wed 16:00-17:30}
\instructor{address}{University of Massachusetts Boston, 100 Morrissey Blvd., Boston, MA}

\usepackage{\texDirectory/template/style/beamerthemeUmassLecture}
\doc{number}{11}
%\setbeamertemplate{footline}[text line]{}

\begin{document}
\prepareCover

\section{Course Administration}

\begin{slide}
	\begin{itemize}
		\item[] Assignment 3 released. Due on November 4, 2015 at 5:30 PM.
	\end{itemize}
\end{slide}

\begin{slide}
	\begin{block}{Overview}
		\begin{itemize}
			\item[] Static Methods
		\end{itemize}
	\end{block}
\end{slide}

\section{Static Methods}

\begin{slide}
	\begin{block}{Definition}
		A method is a self-contained collection of statements that are grouped together to accomplish a specific task.
		Methods are modules that often \textit{take in} data, process it and often \textit{return} an output.
	\end{block}
\end{slide}

\begin{slide}
	\begin{block}{\texttt{HelloWorld.java} Revisited}
		\begin{minted}[fontsize=\small,tabsize=8, linenos, firstnumber=1]{java}
			public class HelloWorld {
			    public static void main(String[] args) {
			        System.out.println("Hello World!");
			    }
			}
		\end{minted}
	\end{block}
	\begin{block}{Anatomy of \texttt{HelloWorld.java}}
		\begin{itemize}
			\item[] One Method \texttt{void main(String[] args)}
			\item[] Not all programs have only one method.
		\end{itemize}
	\end{block}
\end{slide}

\begin{slide}
	\begin{block}{Structure}
		\begin{minted}[fontsize=\small,tabsize=8, linenos, firstnumber=1]{java}
			public static void main(String[] args) {
			    System.out.println("Hello World!");
			}
		\end{minted}
		Method declarations have six components:
		\begin{columns}
			\begin{column}{0.3\textwidth}
				\begin{itemize}
					\item[] modifier
					\item[] \emph{return type}
					\item[] \emph{method name}
					\item[] \emph{parameter list}
					\item[] exception list
					\item[] \emph{method body}
				\end{itemize}
			\end{column}
			\begin{column}{0.7\textwidth}
				\begin{itemize}
					\item[] \texttt{public}
					\item[] \texttt{void}
					\item[] \texttt{main}
					\item[] \texttt{String[] args}
					\item[] -
					\item[] \texttt{\footnotesize \{ System.out.println(``Hello World!''); \}}
				\end{itemize}
			\end{column}
		\end{columns}
	\end{block}
\end{slide}

\begin{slide}
	\begin{block}{Objective}
		Write a program \texttt{Temperature.java} that takes temperature in Celsius as command-line argument and prints it's equivalent in fahrenheit.
	\end{block}
\end{slide}

\begin{slide}
	\begin{block}{\texttt{Temperature.java} (v1)}
		\begin{minted}[fontsize=\small,tabsize=8, linenos, firstnumber=1]{java}
			public class Temperature {
			    public static void main(String[] args) {
			        double celsius = Double.parseDouble(args[0]);
			        double fahrenheit = celsius * 1.8 + 32;
			        System.out.println("%.2f F", fahrenheit);
			    }
			}
		\end{minted}
	\end{block}
\end{slide}

\begin{slide}
	\begin{block}{\texttt{Temperature.java} (v2)}
		\begin{minted}[fontsize=\small,tabsize=8, linenos, firstnumber=1]{java}
			public class Temperature {
			    public static void main(String[] args) {
			        double celsius = Double.parseDouble(args[0]);
			        double fahrenheit = toFahrenheit(celsius);
			        System.out.println("%.2f F", fahrenheit);
			    }
			    public static double toFahrenheit(double celsius) {
			        double fahrenheit = celsius * 1.8 + 32;
			        return fahrenheit;
			    }
			}
		\end{minted}
	\end{block}
\end{slide}

\begin{slide}
	\begin{block}{Motivation}
		\begin{itemize}
			\item[] Modularization
			\item[] Code Reuse
			\item[] Code Test
			\item[] Namespace Organization
		\end{itemize}
	\end{block}
\end{slide}

\begin{slide}
	\begin{block}{Objective}
		Write a program \texttt{PrimeNumber.java} that prints all first 1000 prime numbers.
	\end{block}
\end{slide}

\begin{slide}
	\begin{block}{PrimeNumber.java (v3) (Part 1)}
		\begin{minted}[fontsize=\small,tabsize=8, linenos, firstnumber=1]{java}
			public class PrimeNumber {
			    public static void main(String[] args) {
			        int counter = 0;
			        int num = 2;
			        while (counter < 1000) {
			            if (isPrime(num))
			                counter++;
			            num++;
			        }
			    }
		\end{minted}
	\end{block}
\end{slide}

\begin{slide}
	\begin{block}{PrimeNumber.java (v3) (Part 2)}
		\begin{minted}[fontsize=\small,tabsize=8, linenos, firstnumber=11]{java}
			    public static boolean isPrime(int number) {
			        boolean isPrime = true;
			        for (int i = 2; i <= number / 2; i++)
			            if (number % i == 0) {
			                isPrime = false;
			                break;
			            }
			        return isPrime;
			    }
			}
		\end{minted}
	\end{block}
\end{slide}

\begin{slide}
	\begin{block}{Remember}
		\begin{itemize}
			\item[] All variables defined inside methods have a \emph{local} scope.
		\end{itemize}
	\end{block}
\end{slide}

\begin{slide}
	\begin{block}{Problem Statment}
		\begin{itemize}
			\item[] Non-prime numbers are unnecessarily checked in method \texttt{boolean isPrime(int number)}.
			\item[] It suffices to loop over already-found prime numbers less than $N / 2$.
		\end{itemize}
	\end{block}
	\begin{block}{Proposed Solution}
		\begin{itemize}
			\item[] Use an array \texttt{int[] primes} with size 1000 in method \texttt{boolean isPrime(int number)}.
			\item[] Update \texttt{int[] primes} every time a prime number is found.
		\end{itemize}
	\end{block}
\end{slide}

\section{Static Variables}

\begin{slide}
	\begin{block}{Definition}
		Static variables are variables that belong to the class.
		Their scope is not limited to class methods and is determined by their \textit{modifier}.
	\end{block}
\end{slide}

\begin{slide}
	\begin{block}{Objective}
		Write a program \texttt{PrimeNumber.java} that prints all first 100000 prime numbers.
	\end{block}
\end{slide}

\begin{slide}
	\begin{block}{PrimeNumber.java (v4) (Part 1)}
		\begin{minted}[fontsize=\small,tabsize=8, linenos, firstnumber=1]{java}
			public class PrimeNumber {
			    public static int counter = 0;
			    public static int[] primes = new int[100000];
			    public static void main(String[] args) {
			        int num = 2;
			        while (counter < 100000) {
			            if (isPrime(num)) {
			                System.out.printf("%d, ", num);
			                primes[counter] = num;
			                counter++;
			            }
			            num++;
			        }
			    }
		\end{minted}
	\end{block}
\end{slide}

\begin{slide}
	\begin{block}{PrimeNumber.java (v4) (Part 2)}
		\begin{minted}[fontsize=\small,tabsize=8, linenos, firstnumber=15]{java}
			    public static boolean isPrime(int number) {
			        boolean isPrime = true;
			        for (int i = 0; i < counter; i++)
			            if (number % primes[i] == 0) {
			                isPrime = false;
			                break;
			            }
			        return isPrime;
			    }
			}
		\end{minted}
	\end{block}
\end{slide}

\plain{}{Keep Calm\\and\\Practice}

\end{document}
