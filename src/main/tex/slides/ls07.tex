%%%%%%%%%%%%%%%%%%%%%%%%%%%%%%%%%%%%%%%%%%%%%%%%%%%%%%%%%%%%%%%%%%%%%%
% UMB-CS114-2015F: Introduction to Programming in Java
% Copyright 2015 Pejman Ghorbanzade <pejman@ghorbanzade.com>
% Creative Commons Attribution-ShareAlike 4.0 International License
% More info: https://github.com/ghorbanzade/UMB-CS114-2015F
%%%%%%%%%%%%%%%%%%%%%%%%%%%%%%%%%%%%%%%%%%%%%%%%%%%%%%%%%%%%%%%%%%%%%%

\def \topDirectory {../..}
\def \texDirectory {\topDirectory/src/main/tex}

\documentclass[10pt, compress]{beamer}

\usepackage{\texDirectory/template/style/directives}
%%%%%%%%%%%%%%%%%%%%%%%%%%%%%%%%%%%%%%%%%%%%%%%%%%%%%%%%%%%%%%%%%%%%%%%%%%%%%%
% CS114: Introduction to Programming in Java
% Copyright 2015 Pejman Ghorbanzade <mail@ghorbanzade.com>
% Creative Commons Attribution-ShareAlike 4.0 International License
% https://github.com/ghorbanzade/UMB-CS114-2015F/blob/master/LICENSE
%%%%%%%%%%%%%%%%%%%%%%%%%%%%%%%%%%%%%%%%%%%%%%%%%%%%%%%%%%%%%%%%%%%%%%%%%%%%%%

\course{id}{CS114}
\course{name}{Introduction to Java}
\course{venue}{Mon/Wed, 5:30 PM - 6:45 PM}
\course{semester}{Fall 2015}
\course{department}{Department of Computer Science}
\course{university}{University of Massachusetts Boston}

\instructor{name}{Pejman Ghorbanzade}
\instructor{title}{}
\instructor{position}{Student Instructor}
\instructor{email}{pejman@cs.umb.edu}
\instructor{phone}{617-287-6419}
\instructor{office}{S-3-124B}
\instructor{office-hours}{Mon/Wed 16:00-17:30}
\instructor{address}{University of Massachusetts Boston, 100 Morrissey Blvd., Boston, MA}

\usepackage{\texDirectory/template/style/beamerthemeUmassLecture}
\doc{number}{7}
%\setbeamertemplate{footline}[text line]{}

\begin{document}
\prepareCover

\section{Course Administration}

\begin{slide}
	\begin{itemize}
		\item[] Assignment 1 Due October 5, 2015 at 17:30 PM.
	\end{itemize}
\end{slide}

\begin{slide}
	\begin{block}{Overview}
		\begin{itemize}
			\item[] Class Scanner
			\item[] Conditionals
		\end{itemize}
	\end{block}
\end{slide}

\section{Class Scanner}

\begin{slide}
	\begin{block}{Problem Definition}
		\begin{itemize}
			\item[] Command-line arguments to be given before program execution.
			\begin{itemize}
				\item[] What if we want to get input when certain conditions are met?
			\end{itemize}
		\end{itemize}
	\end{block}
	\begin{block}{Solution}
		\begin{itemize}
			\item[] Scanner class provides a simple yet powerful way to ask for inputs during execution of program.
		\end{itemize}
	\end{block}
\end{slide}

\begin{slide}
	\begin{block}{Objective}
		Write a program \texttt{Greetings.java} that prompts user for his name and then prints a welcome message of the following format.
	\end{block}
	\begin{block}{Runtime Example}
		\begin{minted}[fontsize=\small,tabsize=8]{text}
			$ javac Greetins
			$ java Greetings
			Hello! What is your name? Pejman
			Greetings, Pejman!
		\end{minted}
	\end{block}
\end{slide}

\begin{slide}
	\begin{block}{Example}
		\begin{minted}[fontsize=\small,tabsize=8]{java}
			import java.util.Scanner;
			public class Greetings {
			    public static void main(String[] args) {
			        Scanner input = new Scanner(System.in);
			        System.out.print("Hello! What is your name? ");
			        String userName = input.nextLine();
			        input.close();
			        System.out.println("Greetings, " + userName + "!");
			    }
			}
		\end{minted}
	\end{block}
\end{slide}

\begin{slide}
	\begin{block}{Objective}
		Write a program \texttt{SimpleAdder.java} that prompts user for two \textit{Integer} numbers and shows their sum as output.
	\end{block}
	\begin{block}{Runtime Example}
		\begin{minted}[fontsize=\small,tabsize=8]{text}
			$ javac SimpleAdder
			$ java SimpleAdder
			a = ? 124
			b = ? 155
			a + b = 279
		\end{minted}
	\end{block}
\end{slide}

\begin{slide}
	\begin{block}{Example}
		\begin{minted}[fontsize=\small,tabsize=8]{java}
			import java.util.Scanner;
			public class SimpleAdder {
			    public static void main(String[] args) {
			        Scanner input = new Scanner(System.in);
			        System.out.print("a = ? ");
			        int num1 = input.nextInt();
			        System.out.print("b = ? ");
			        int num2 = input.nextInt();
			        input.close();
			        System.out.println("a + b = " + (num1 + num2));
			    }
			}
		\end{minted}
	\end{block}
\end{slide}

\begin{slide}
	\begin{block}{Instantiation}
		\begin{minted}[fontsize=\small,tabsize=8]{java}
			Scanner input = new Scanner(System.in);
		\end{minted}
	\end{block}
	\begin{block}{Methods}
		\begin{columns}
			\begin{column}{0.5\textwidth}
				\begin{itemize}
					\item[] reset()
					\item[] skip()
					\item[] match()
					\item[] useDelimiter()
					\item[] close()
				\end{itemize}
			\end{column}
			\begin{column}{0.5\textwidth}
				\begin{itemize}
					\item[] next()
					\item[] nextLine()
					\item[] nextInt()
					\item[] nextDouble()
					\item[] hasNext()
				\end{itemize}
			\end{column}
		\end{columns}
	\end{block}
\end{slide}

\begin{slide}
	\begin{block}{Final Note}
		\begin{itemize}
			\item[] It is a good practice to use the \texttt{close()} method to close once you are sure you will no longer use it. This helps avoid memory leakage and releases system resources.
			\item[] Once \texttt{Scanner} object is closed, you will no longer be able to ask for inputs from \texttt{System.in}. To avoid this issue, you must have another Scanner object on which \texttt{close()} method is not invoked yet.
		\end{itemize}
	\end{block}
	\begin{block}{Example}
		\begin{minted}[fontsize=\small,tabsize=8]{java}
			input.close();
		\end{minted}
	\end{block}
\end{slide}

\section{Conditionals}

\begin{slide}
	\begin{block}{Objective}
		Write a program \texttt{Quadratic.java} that takes three integers \texttt{a}, \texttt{b} and \texttt{c} as command line arguments and solves for \texttt{x} with three digits of precision the quadratic expression shown below.
		\begin{equation*}\label{eq1}
			a^2 x + b x + c = 0
		\end{equation*}
	\end{block}
\end{slide}

\begin{slide}
	\begin{block}{\texttt{Quadratic.java} (v1)}
		\begin{minted}[fontsize=\small,tabsize=8, linenos, firstnumber=1]{java}
			public class Quadratic {
			    public static void main(String[] args) {
			        double a = Double.parseDouble(args[0]);
			        double b = Double.parseDouble(args[1]);
			        double c = Double.parseDouble(args[2]);
			        double discriminant = Math.pow(b,2) - 4 * a * c;
			        double sol1 = (- b + Math.sqrt(discriminant))/(2*a);
			        double sol2 = (- b - Math.sqrt(discriminant))/(2*a);
			        System.out.printf("Solutions are %.3f and %.3f\n",
			            sol1,sol2);
			    }
			}
		\end{minted}
	\end{block}
\end{slide}

\begin{slide}
	\begin{block}{Problem Statement}
		\begin{minted}[fontsize=\small,tabsize=8]{text}
			$ javac Quadratic.java
			$ java Quadratic 1 -2 -4
			Solutions are -1.236 and 3.237
			$ java Quadratic 9 12 4
			Solutions are -0.667 and -0.667
			$ java Quadratic 3 4 2
			Solutions are NaN and NaN
		\end{minted}
	\end{block}
	\begin{block}{Proposed Solution}
		What if we could make the program decide what to print based on value of discriminant?
	\end{block}
\end{slide}

\begin{slide}
	\begin{block}{\texttt{Quadratic.java} (v2)}
		\begin{minted}[fontsize=\small,tabsize=8, linenos, firstnumber=3]{java}
			double a = Double.parseDouble(args[0]);
			double b = Double.parseDouble(args[1]);
			double c = Double.parseDouble(args[2]);
			double discriminant = Math.pow(b,2) - 4 * a * c;
			if (discriminant > 0 ) {
			    double sol1 = (- b + Math.sqrt(discriminant))/(2*a);
			    double sol2 = (- b - Math.sqrt(discriminant))/(2*a);
			    System.out.printf("Solutions are %.3f and %.3f\n",
			        sol1, sol2);
			}
		\end{minted}
	\end{block}
\end{slide}

\begin{slide}
	\begin{block}{Problem Statement}
		\begin{minted}[fontsize=\small,tabsize=8]{text}
			$ javac Quadratic.java
			$ java Quadratic 1 -2 -4
			Solutions are -1.236 and 3.237
			$ java Quadratic 9 12 4
			$ java Quadratic 3 4 2
		\end{minted}
	\end{block}
	\begin{block}{Proposed Solution}
		Use more \texttt{if} conditionals to cover different conditions.
	\end{block}
\end{slide}

\begin{slide}
	\begin{block}{\texttt{Quadratic.java} (v3)}
		\begin{minted}[fontsize=\small,tabsize=8, linenos, firstnumber=6]{java}
			double discriminant = Math.pow(b,2) - 4 * a * c;
			if (discriminant > 0 ) {
			    double sol1 = (- b + Math.sqrt(discriminant)) / (2 * a);
			    double sol2 = (- b - Math.sqrt(discriminant)) / (2 * a);
			    System.out.printf("Solutions are %.3f and %.3f\n",
			        sol1, sol2);
			}
			if (discriminant == 0) {
			    double sol1 = (-b) / (2 * a);
			    System.out.printf("Solution is %.3f\n", sol1);
			}
			if (discriminant < 0)
			    System.out.printf("No real solution exists.");
		\end{minted}
	\end{block}
\end{slide}

\begin{slide}
	\begin{block}{Problem Statement}
		\begin{minted}[fontsize=\small,tabsize=8]{text}
			> javac Quadratic.java
			> java Quadratic 1 -2 -4
			Solutions are -1.236 and 3.237
			> java Quadratic 9 12 4
			Solution is -0.667
			> java Quadratic 3 4 2
			No real solution exists.
		\end{minted}
	\end{block}
	\begin{block}{Are we done?}
		All three conditions are being checked unnecessarily.
	\end{block}
\end{slide}

\begin{slide}
	\begin{block}{\texttt{Quadratic.java} (v4)}
		\begin{minted}[fontsize=\small,tabsize=8, linenos, firstnumber=6]{java}
			double discriminant = Math.pow(b,2) - 4 * a * c;
			if (discriminant > 0 ) {
			    double sol1 = (- b + Math.sqrt(discriminant)) / (2 * a);
			    double sol2 = (- b - Math.sqrt(discriminant)) / (2 * a);
			    System.out.printf("Solutions are %.3f and %.3f\n",
			        sol1, sol2);
			}
			else if (discriminant == 0) {
			    double sol1 = (-b) / (2 * a);
			    System.out.printf("Solution is %.3f\n", sol1);
			}
			else (discriminant < 0)
			    System.out.printf("No real solution exists.");
		\end{minted}
	\end{block}
\end{slide}

\begin{slide}
	\begin{block}{Objective}
		Write a program \textit{NumberChecker.java} that prompts user for an \textit{Integer} number and determines whether the number can be stored in a variable of type \textit{Byte} or \textit{Short}.
	\end{block}
\end{slide}

\begin{slide}
	\begin{block}{\texttt{NumberChecker.java} (v1)}
		\begin{minted}[fontsize=\small,tabsize=8, linenos]{java}
			import java.util.Scanner;
			public class NumberChecker {
			    public static void main(String[] args) {
			        Scanner input = new Scanner(System.in);
			        System.out.print("Enter number? ");
			        int number = input.nextInt();
			        if (number <= Byte.MAX_VALUE)
			            System.out.printf("%d fits in Byte.\n", number);
			        if (number <= Short.MAX_VALUE)
			            System.out.print(number + " fits in Short.\n");
			        if (number <= Integer.MAX_VALUE)
			            System.out.println(number + " fits in Integer.");
			    }
			}
		\end{minted}
	\end{block}
\end{slide}

\begin{slide}
	\begin{block}{Problem Statement}
		\begin{minted}[fontsize=\small,tabsize=8]{text}
			$ java NumberChecker
			Enter number? 12345678
			12345678 fits in Integer.
			$ java NumberChecker
			Enter number? 123
			123 fits in Byte
			123 fits in Short
			123 fits in Integer
		\end{minted}
	\end{block}
	\begin{block}{Proposed Solution}
		Condition to be checked if previous condition is not met.
	\end{block}
\end{slide}

\begin{slide}
	\begin{block}{\texttt{NumberChecker.java} (v2)}
		\begin{minted}[fontsize=\small,tabsize=8, linenos]{java}
			import java.util.Scanner;
			public class NumberChecker {
			    public static void main(String[] args) {
			        Scanner input = new Scanner(System.in);
			        System.out.print("Enter number? ");
			        int number = input.nextInt();
			        if (number <= Byte.MAX_VALUE)
			            System.out.printf("%d fits in Byte.\n", number);
			        elseif (number <= Short.MAX_VALUE)
			            System.out.print(number + " fits in Short.\n");
			        else
			            System.out.println(number + " fits in Integer.");
			    }
			}
		\end{minted}
	\end{block}
\end{slide}

\begin{slide}
	\begin{block}{Result}
		\begin{minted}[fontsize=\small,tabsize=8]{text}
			$ java NumberChecker
			Enter number? 12345
			12345 fits in Short.
			$ java NumberChecker
			Enter number? 123
			123 fits in Byte.
		\end{minted}
	\end{block}
	\begin{block}{Conclusion}
		Conditionals allow a program to perform different operations in different conditions.
	\end{block}
\end{slide}

\begin{slide}
	\begin{block}{Objective}
		Write a program \texttt{Grade.java} that takes a letter grade and converts it to its numerical equivalent, based on the table given below.
		\begin{table}[H]
			\begin{tabular}{c|c}
				Letter Grade & Numerical Grade\\
				\hline
				A & 4\\
				B & 3\\
				C & 2\\
				D & 1\\
				E & 0\\
			\end{tabular}
		\end{table}
	\end{block}
\end{slide}

\begin{slide}
	\begin{block}{\texttt{Grade.java} (v1.0)}
		\begin{minted}[fontsize=\small,tabsize=8, linenos, firstnumber=4]{java}
			        Scanner input = new Scanner(System.in);
			        System.out.print("Insert a letter grade: ");
			        String letter = input.nextLine();
			        if (letter.equals("A"))
			            System.out.println("GPA is 4");
			        else if (letter.equals("B"))
			            System.out.println("GPA is 3");
			        else if (letter.equals("C"))
			            System.out.println("GPA is 2");
			        else if (letter.equals("D"))
			            System.out.println("GPA is 1");
			        else
			            System.out.println("GPA is 0");
		\end{minted}
	\end{block}
\end{slide}

\begin{slide}
	\begin{block}{Problem Statement}
		We were lucky grade letters end in \texttt{F}.
	\end{block}
	\begin{block}{Disagree?}
		Write a program \texttt{Month.java} that takes a number between 1 to 12 and prints full name of the month of the year corresponding to that number.
	\end{block}
	\begin{block}{Proposed Solution}
		\texttt{switch} conditionals help us avoid repetitive \texttt{if-then-else} conditionals.
	\end{block}
\end{slide}

\begin{slide}
	\begin{block}{\texttt{Grade.java} (v2.0)}
		\begin{minted}[fontsize=\small,tabsize=8, linenos, firstnumber=3]{java}
			        Scanner input = new Scanner(System.in);
			        System.out.print("Insert a letter grade: ");
			        String letter = input.nextLine();
			        input.close();
			        double num;
			        switch (letter.charAt(0)) {
			            case 'A': num = 4; break;
			            case 'B': num = 3; break;
			            case 'C': num = 2; break;
			            case 'D': num = 1; break;
			            default: num = 0; break;
			        }
			        System.out.printf("GPA is %.2f\n", num);
		\end{minted}
	\end{block}
\end{slide}

\begin{slide}
	\begin{block}{Remember}
		\begin{itemize}
			\item[] \texttt{switch} is not always an alternative for the \texttt{if-then-else} conditional. \texttt{switch} is a very special case of more general \texttt{if-then-else} conditionals.
			\item[] It is always necessary to use \texttt{break} for cases when we want to prevent other cases to be checked.
			\item[] The \texttt{default} section handles all values that are not explicitly handled by one of the \texttt{case} sections.
		\end{itemize}
	\end{block}
\end{slide}

\plain{}{Keep Calm\\and\\Practice}

\end{document}
