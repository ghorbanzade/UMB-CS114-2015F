\section*{Question 1}

An $n \times n$ matrix is called a \textit{positive Markov matrix} if each element is positive and the sum of the elements in each column is 1.
Write a program \texttt{MarkovMatrix.java} that prompts the user to enter a $3 \times 3$ matrix of double values.
Use a method with the following signature to test if the given matrix is a Markov matrix.

\begin{terminal}
public static boolean isMarkovMatrix(double[][] matrix)
\end{terminal}

Your program is expected to function as shown in following examples:

\begin{terminal}
$ javac MarkovMatrix.java
$ java MarkovMatrix
Enter Row 1: 0.15 0.875 0.375
Enter Row 2: 0.55 0.005 0.225
Enter Row 3: 0.30 0.12 0.4
Markov matrix given.
\end{terminal}

\section*{Question 2}
Write a program \texttt{PointAndSphere2.java} that prompts user to enter respectively, coordinates of a point, coordinates of the center of a sphere and radius of the sphere.
Your program would then determine the location of the point with respect to the sphere.
The point should be an instance of class \texttt{Point} and the sphere should be an instance of class \texttt{Sphere}.
Following is an example of an accepted output format.

\vfill
\newpage

\begin{terminal}
$ javac Point.java Sphere.java Coordinate.java PointAndSphere2.java
$ java PointAndSphere2
Coordinates of Point: 1 1 1
Coordinates of Sphere: 0 0 0
Radius of Sphere: 1.7
The point is outside the sphere.
\end{terminal}

\section*{Question 3}
Write a program \texttt{MatrixFiller2.java} that prompts user for a number $x$ between 1 to 9 and instantiates an $x \times x$ matrix from class \texttt{Matrix.java} whose elements are randomly generated from range 1 to $x^2$.
Following is an expected sample run of your program.

\begin{terminal}
$ javac Matrix.java MatrixFiller2.java
$ java MatrixFiller2
Size of Matrix: 4
 05 13 07 16
 12 02 10 01
 09 14 14 08
 02 05 01 14
\end{terminal}

\section*{Question 4}

Write a class \texttt{Circle.java} from which we can instantiate a circle by giving its radius and use it as is shown in the following program.

\lstset{caption=}
\lstset{language=Java,tabsize=4}
\begin{lstlisting}
import java.util.Scanner;
public class Circles {
	public static void main(String[] args) {
		Scanner input = new Scanner(System.in);
		System.out.print("Enter radius: ");
		double radius = input.nextDouble();
		input.close();
		Circle myCircle = new Circle(radius);
		double area = myCircle.getArea();
		double perimeter = myCircle.getCircumference();
		System.out.printf("Area: %.2f, Perimeter: %.2f\n", area, perimeter);
	}
}
\end{lstlisting}

\section*{Question 5}
The code snippet given below is content of a file \texttt{Kitten.java} found in a public repository.
Unfortunately, the program cannot be executed because the file \texttt{Cat.java} which defines the class \texttt{Cat} is missing.
You are expected to develop the class \texttt{Cat} in a file \texttt{Cat.java} such that \texttt{Kitten.java} is successfully executed.

\lstset{caption=}
\lstset{language=Java,tabsize=4}
\begin{lstlisting}
import java.util.Scanner;
public class Kitten {
	public static void main(String[] args) {
		Cat myCat = new Cat("Kitty");
		double[] movement = promptMove(myCat);
		myCat.move(movement[0], movement[1]);
		myCat.showPosition();
		myCat.showDistance();
	}
	public static double[] promptMove(Cat myCat) {
		Scanner input = new Scanner(System.in);
		char[] directions = {'X', 'Y'};
		double[] movement = new double[directions.length];
		for (int i = 0; i < directions.length; i++) {
			System.out.printf("Distance to move in %c direction: ", directions[i]);
			movement[i] = input.nextDouble();
		}
		input.close();
		return movement;
	}
}
\end{lstlisting}

Following is a sample expected run of the program.

\begin{terminal}
$ javac Cat.java Kitten.java
$ java Kitten
Distance to move in X direction: 3
Distance to move in Y direction: 4
Kitty is in (3.0, 4.0).
Kitty is 5.00 units away from (0, 0).
\end{terminal}
